%%%%%%%%%%%%%%%%%%%%%%%%%%%%%%%%%%%%%%%%%%%%%%%%%%%%%%%%%%%%%%%%%%%%%%%%%%%%%%%%%%%%%%%%%%%%%%%%%%%%%%%%%%%%%%%%%%%%%%%%%%%%%%%%%%%%%%%%%%%%%%%%%%%%%%%%%%%%%%%%%%%%%%%
%%%%%%%%%%%%%%%%%%%%%%%%%%%%%%%%%%%%%%%%%%%%%%%%%%%%%%%%%%%%%%%%%%%%%%%%%%%%%%%%%%%%%%%%%%%%%%%%%%%%%%%%%%%%%%%%%%%%%%%%%%%%%%%%%%%%%%%%%%%%%%%%%%%%%%%%%%%%%%%%%%%%%%%
%%% Modèle pour la 4ème de couverture des thèses préparées à l'Université Paris-Saclay, basé sur le modèle produit par Nikolas STOTT / Template for back cover of thesis made at Université Paris-Saclay, based on the template made by Nikolas STOTT
%%% Mis à jour par Aurélien ARNOUX (École polytechnique)/ Updated by Aurélien ARNOUX (École polytechnique)
%%% Les instructions concernant chaque donnée à remplir sont données en bloc de commentaire / Rules to fill this file are given in comment blocks
%%% ATTENTION Ces informations doivent tenir sur une seule page une fois compilées / WARNING These informations must contain in no more than one page once compiled
%%%%%%%%%%%%%%%%%%%%%%%%%%%%%%%%%%%%%%%%%%%%%%%%%%%%%%%%%%%%%%%%%%%%%%%%%%%%%%%%%%%%%%%%%%%%%%%%%%%%%%%%%%%%%%%%%%%%%%%%%%%%%%%%%%%%%%%%%%%%%%%%%%%%%%%%%%%%%%%%%%%%%%%
%%% Version: 2022-08-02 @author : riverarodrigoa (based on the template of bmazoyer)
%%%%%%%%%%%%%%%%%%%%%%%%%%%%%%%%%%%%%%%%%%%%%%%%%%%%%%%%%%%%%%%%%%%%%%%%%%%%%%%%%%%%%%%%%%%%%%%%%%%%%%%%%%%%%%%%%%%%%%%%%%%%%%%%%%%%%%%%%%%%%%%%%%%%%%%%%%%%%%%%%%%%%%%


\label{form_last}
%%%%%%%%%%%%%%%%%%%%%%%%%%%%%%%%%%%%%%%%%%%%%%%%%%%%%%%%%%%%%%%%%%%%%%%%%%%%%%%%%%%%%%%%%%%%%%%%%%%%%%%%%%%%%%%%%%%%%%%%%%%%%%%%%%%%%%%%%%%%%%%%%%%%%%%%%%%%%%%%%%%%%%%
%%%%%%%%%%%%%%%%%%%%%%%%%%%%%%%%%%%%%%%%%%%%%%%%%%%%%%%%%%%%%%%%%%%%%%%%%%%%%%%%%%%%%%%%%%%%%%%%%%%%%%%%%%%%%%%%%%%%%%%%%%%%%%%%%%%%%%%%%%%%%%%%%%%%%%%%%%%%%%%%%%%%%%%
%%% Formulaire / Form
%%% Remplacer les paramètres des \newcommand par les informations demandées / Replace \newcommand parameters by asked informations
%%%%%%%%%%%%%%%%%%%%%%%%%%%%%%%%%%%%%%%%%%%%%%%%%%%%%%%%%%%%%%%%%%%%%%%%%%%%%%%%%%%%%%%%%%%%%%%%%%%%%%%%%%%%%%%%%%%%%%%%%%%%%%%%%%%%%%%%%%%%%%%%%%%%%%%%%%%%%%%%%%%%%%%
%%%%%%%%%%%%%%%%%%%%%%%%%%%%%%%%%%%%%%%%%%%%%%%%%%%%%%%%%%%%%%%%%%%%%%%%%%%%%%%%%%%%%%%%%%%%%%%%%%%%%%%%%%%%%%%%%%%%%%%%%%%%%%%%%%%%%%%%%%%%%%%%%%%%%%%%%%%%%%%%%%%%%%%

\newcommand{\logoEd}{ABIES_2}																		%% Logo de l'école doctorale. Indiquer le sigle / Doctoral school logo. Indicate the acronym : 2MIB; AAIF; ABIES; BIOSIGNE; CBMS; EDMH; EDOM; EDPIF; EDSP; EOBE; INTERFACES; ITFA; PHENIICS; SDSV; SDV; SHS; SMEMAG; SSMMH; STIC
\newcommand{\PhDTitleFR}{Modéliser la crise du vivant : les rôles de l'espace et des comportements stratégiques dans la modélisation bioéconomique}													%% Titre de la thèse en français / Thesis title in french
\newcommand{\keywordsFR}{Modélisation bioéconomique; Biodiversité; Connectivité Spatiale; Exploitation et Contrôle; Equilibre non coopératif; Politique Publique}														%% Mots clés en français, séprarés par des , / Keywords in french, separated by ,
\newcommand{\abstractFR}{

La biodiversité mondiale subit des pressions croissantes dues aux activités humaines : perte d'habitats, surexploitation des ressources, changements climatiques, introduction d'espèces invasives. Ces menaces affectent non seulement la diversité des espèces, mais aussi les services écosystémiques essentiels. Les politiques de conservation, malgré les initiatives internationales et nationales, ont des performances mitigées. \\

Comment mieux concevoir des politiques pour freiner l’effondrement de la biodiversité tout en prenant en compte les enjeux économiques ? Comment intégrer l’espace à l’analyse économique de la biodiversité? Comment intégrer les interactions entre agents stratégiques autour de la biodiversité ? \\

Cette thèse modélise la perte et fragmentation de l’habitat ainsi que la surexploitation (ou le sous contrôle) des espèces en utilisant une approche bioéconomique, qui combine l’économie d’une part et l’écologie des populations et des paysages d’autre part. Elle met l’accent sur le rôle de l’espace et les interactions stratégiques, afin d’orienter les politiques publiques dans divers écosystèmes.\\
Le premier chapitre passe en revue la littérature bioéconomique sur les systèmes socio-écologiques terrestres, et identifie deux grands paradigmes : l'optimisation économique des ressources et la conservation de la biodiversité dans des paysages aménagés. Il expose les défis méthodologiques de la modélisation bioéconomique pour mieux saisir la crise de la biodiversité, et apporter des pistes de solutions.\\
Le deuxième chapitre développe un modèle de gestion des paysages forestiers confrontés au dilemme entre conservation
et réduction des risques d’incendie. En utilisant la théorie des graphes, il définit une frontière de production entre ces objectifs sous contrainte budgétaire et caractérise les localisation des opérations de traitement des combustibles optimales, ainsi que les paysages qui en découlent.\\
Le troisième chapitre traite de la gestion des espèces nuisibles en intégrant l’usage de clôtures écologiques. Le modèle explore comment ces barrières limitent la propagation d’espèces nuisibles dans un réseau de parcelles, en intégrant des coûts de contrôle hétérogènes. Il considère la question des externalités spatiales, et de la gestion des opportunités d’arbitrage spatiales. Il propose une analyse de la connectivité des paysages comme décision et examine la gestion optimale et non coopérative des espèces nuisibles mobiles. \\
Le quatrième chapitre analyse l’exploitation du Totoaba macdonaldi, un poisson en danger critique à cause du braconnage d’un cartel. Il revisite un modèle bioéconomique et analyse les conséquences d’un monopole vertical sur le totoaba. Il montre ensuite que l’aquaculture peut réduire la pression sur les populations sauvages, contrairement à l'idée que celle-ci intensifie l'exploitation. L’élevage de totoaba stabilise les populations et réduit les incitations au braconnage, avec des implications importantes pour la conservation.\\
En conclusion, cette thèse propose des développements dans la modélisation bioéconomique, en prenant en compte l’espace et les comportements stratégiques, pour analyser la crise de la biodiversité, ouvrant la voie à des politiques publiques plus efficaces conciliant conservation et développement économique.
}															%% Résumé en français / abstract in french

\newcommand{\PhDTitleEN}{Modeling the biodiversity crisis : the roles of space and strategic behavior in bioeconomic modeling}													%% Titre de la thèse en anglais / Thesis title in english
\newcommand{\keywordsEN}{Bioeconomic modeling; Biodiversity; Spatial Connectivity; Harvest and control; Non cooperative equilibrium; Public policy}														%% Mots clés en anglais, séprarés par des , / Keywords in english, separated by ,
\newcommand{\abstractEN}{

The world's biodiversity is under increasing pressure from human activities: habitat loss, overexploitation of resources, climate change, and introduction of invasive species. These threats affect not only species diversity but also essential ecosystem services. Despite international and national initiatives, conservation policies have had mixed results.\\

 How can we better design policies to halt the collapse of biodiversity while taking economic issues into account? How can we integrate space in the economic analysis of biodiversity ? How can we integrate the interactions between strategic agents around biodiversity? \\

This thesis models habitat loss and fragmentation, as well as the overexploitation (or undercontrol) of species, using a bioeconomic approach that combines economics on the one hand, and population and landscape ecology on the other. It focuses on the role of space and strategic interactions in order to guide public policies in various ecosystems.\\

The first chapter reviews the bioeconomic literature on terrestrial social-ecological systems and identifies two major paradigms: economic optimization of resources and biodiversity conservation in managed landscapes. It outlines the methodological challenges of bioeconomic modeling to better grasp the biodiversity crisis and provide possible solutions. \\

The second chapter develops a management model for forest landscapes faced with a dilemma between conservation and wildfire risk reduction. Using graph theory, it defines a production frontier between these objectives under budgetary constraints, and characterizes the locations of optimal fuel treatment operations, as well as the resulting landscapes. \\

The third chapter deals with the management of pest species by integrating the use of ecological fences. The model explores how these barriers limit the spread of pest species in a network of plots, integrating heterogeneous control costs. It considers the issue of spatial externalities and the management of spatial arbitrage opportunities. It proposes an analysis of landscape connectivity as a decision and examines optimal and non-cooperative management of spatial public bads. \\

The fourth chapter analyzes the exploitation of Totoaba macdonaldi, a critically endangered fish due to cartel poaching. It revisits a bioeconomic model and analyzes the consequences of a vertical monopoly on totoaba. We then show that aquaculture can reduce pressure on wild populations, contrary to the idea that it intensifies exploitation. Totoaba farming stabilizes populations and reduces incentives for poaching, with important implications for conservation.\\

In conclusion, this thesis proposes developments in bioeconomic modeling, taking into account space and strategic behaviors, to analyze the biodiversity crisis, paving the way for more effective public policies reconciling conservation and economic development.


}															%% Résumé en anglais / abstract in english

\label{layout_last}
%%%%%%%%%%%%%%%%%%%%%%%%%%%%%%%%%%%%%%%%%%%%%%%%%%%%%%%%%%%%%%%%%%%%%%%%%%%%%%%%%%%%%%%%%%%%%%%%%%%%%%%%%%%%%%%%%%%%%%%%%%%%%%%%%%%%%%%%%%%%%%%%%%%%%%%%%%%%%%%%%%%%%%%
%%%%%%%%%%%%%%%%%%%%%%%%%%%%%%%%%%%%%%%%%%%%%%%%%%%%%%%%%%%%%%%%%%%%%%%%%%%%%%%%%%%%%%%%%%%%%%%%%%%%%%%%%%%%%%%%%%%%%%%%%%%%%%%%%%%%%%%%%%%%%%%%%%%%%%%%%%%%%%%%%%%%%%%
%%% Mise en page / Page layout      
%%% NE RIEN MODIFIER / DO NOT MODIFY
%%%%%%%%%%%%%%%%%%%%%%%%%%%%%%%%%%%%%%%%%%%%%%%%%%%%%%%%%%%%%%%%%%%%%%%%%%%%%%%%%%%%%%%%%%%%%%%%%%%%%%%%%%%%%%%%%%%%%%%%%%%%%%%%%%%%%%%%%%%%%%%%%%%%%%%%%%%%%%%%%%%%%%%
%%%%%%%%%%%%%%%%%%%%%%%%%%%%%%%%%%%%%%%%%%%%%%%%%%%%%%%%%%%%%%%%%%%%%%%%%%%%%%%%%%%%%%%%%%%%%%%%%%%%%%%%%%%%%%%%%%%%%%%%%%%%%%%%%%%%%%%%%%%%%%%%%%%%%%%%%%%%%%%%%%%%%%%

\pagestyle{empty}

%%% Logo de l'école doctorale. Le nom du fichier correspond au sigle de l'ED / Doctoral school logo. Filename correspond to doctoral school acronym
%%% Les noms valides sont / Valid names are : 2MIB; AAIF; ABIES; BIOSIGNE; CBMS; EDMH; EDOM; EDPIF; EDSP; EOBE; INTERFACES; ITFA; PHENIICS; SDSV; SDV; SHS; SMEMAG; SSMMH; STIC
\begin{textblock*}{61mm}(16mm,3mm)
    \textblockcolour{white}
	\noindent\includegraphics[height=24mm]{media/ed/\logoEd}
\end{textblock*}



%%%Titre de la thèse en français / Thesis title in french
\begin{singlespace}
\begin{center}
\vspace*{1cm}
%\resizebox{\textwidth}{!}{%
\fcolorbox{bordeau}{white}{\parbox{1.05\textwidth}{%
{\bf Titre:} \PhDTitleFR 
\medskip

%%%Mots clés en français, séprarés par des ; / Keywords in french, separated by ;
{\bf Mots clés:} \keywordsFR 
\vspace{-2mm}

%%% Résumé en français / abstract in french
\begin{multicols}{2}
{\bf Résumé:} 
{\small
\abstractFR 
}
\end{multicols}
}}
%}
\end{center}



%%%Titre de la thèse en anglais / Thesis title in english
\begin{center}
\vspace*{1.5cm}
\resizebox{\textwidth}{!}{%
\fcolorbox{bordeau}{white}{\parbox{0.95\textwidth}{%
{\bf Title:} \PhDTitleEN 

\medskip

%%%Mots clés en anglais, séprarés par des ; / Keywords in english, separated by ;
{\bf Keywords:}  \keywordsEN %%3 à 6 mots clés%%
\vspace{-2mm}
\begin{multicols}{2}
	
%%% Résumé en anglais / abstract in english
{\bf Abstract:} 
{\small
\abstractEN
}
\end{multicols}
}}}
\end{center}

%\begin{textblock*}{161mm}(10mm,270mm)
%\textblockcolour{white}
%\color{bordeau}
%{\bf\noindent Université Paris-Saclay	         }

%\noindent Espace Technologique / Immeuble Discovery 

%\noindent Route de l’Orme aux Merisiers RD 128 / 91190 Saint-Aubin, France 
%\end{textblock*}

%\begin{textblock*}{0mm}(182mm,255mm)
%\textblockcolour{white}
%\includegraphics[width=10mm]{media/UPSACLAY-petit}
%\end{textblock*}
\end{singlespace}