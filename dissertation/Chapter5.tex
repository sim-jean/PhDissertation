\chapter{Fences : the economics of movement in mobile public bads}
\label{chapter3}

\begin{center}

\textbf{Abstract}\par
    \vspace*{.2cm}
    \noindent
    \begin{minipage}{0.9\textwidth}
    \singlespacing
This article examines the management of spatially distributed renewable resources—specifically wildlife and infectious diseases—through the lens of economic and spatial analysis. I focus on "bads" like invasive species and diseases, which cause economic and ecological harm, and utilize population control and fencing as central  mechanisms. I analyze how fencing influences resource flow and connectivity. On the one hand, in the presence of ecological and economic heterogeneities, fencing can be used to leverage spatial artbitrage opportunities. On the other hand, while promoted as a tool to incentivize the internalization of costs associated with ``bads", they may undo what Nature has rightfully done. In this sense, while fencing may be welfare improving in a setting with initially poor connectivity, an uncoordinated use of fencing, although welfare improving, is not welfare maximizing. The study develops a theoretical model that integrates aspects of stock and patch connectivity management and explores both cooperative and non-cooperative management strategies. The findings indicate that optimal management often requires a nuanced understanding of the spatial dynamics and economic costs associated with different control strategies. We present a series of propositions that characterize the conditions under which fencing and resource control strategies can be optimized, including the interaction effects of exclusionary and trap effects. This article contributes to the literature by highlighting the role of spatial heterogeneity in the management of renewable resources and providing insights into the formulation of more effective environmental policies, as it analyzes how to design policies on a subset of the landscape, to maximize economic and ecological benefits. \\\\
\textit{JEL codes :} Q20, Q24, R12\\
\textbf{Keywords :} spatial resource management, invasive species; fencing and control strategies; optimal management; non-cooperative equilibrium; second-best policy. 
\end{minipage}
\end{center}
    \vfill


\newpage


\section{Introduction}
\onehalfspacing
% Motivation paragraph
In the 1600s, the Ma'ohi,  the Indigenous People of the Society Islands in French Polynesia \citep{oliver2019} built vast fish traps, using organic fences, stakes, and poles. On the island of Huahine, stones set vertically, forming V-shaped enclosures trapped schools of fish coming down to sea, from a shallow salt water lake. Fish were pulled towards the sea with the tides, and became trapped in basins. Fish were then harvested using nets in the shallow lake. Managing the fish stock for the community amounted to more than harvesting.  Trapping, thus reducing the extent of fish school mobility, was instrumental\footnote{Modern applications, such as fish fences on Pacific islands, are detrimental to seascape connectivity, and destroy the sea bed, see \cite{exton_artisanal_2019}}.
\\
\textbf{This should ne moved to the conclusion section : works with goods, but also with mixed goods and bads}
\\
Centuries later, in the US populations of white tailed deers have skyrocketted to an estimated 36 million, with exceptionally high densities in the South East \citep{hanberry_regaining_2020}. 
At high densities, deer populations threaten the regeneration of forests as they influence species composition and abundance through browsing, hence damaging people's properties \citep{hanberry_does_2019}. Moreover, risks of zoonosis and epidemics increase with large populations. While large scale culling policies have been implemented, landowners have increasingly resorted to other methods, such as repellents, or fencing. Eight-foot or higher woven-wire fences have been used to protect agricultural land such as orchards as well as private homes, to limit the damage done by growing deer populations \citep{caslick_economic_1979}.
 Eventually, during the COVID 19 pandemic between 2019 and 2023, international airports and ports were shutdown, and extensive lockdown policies were implemented worldwide. By avoiding contact between infected and non-infected people, these policies aimed at slowing the spread of the pandemic\footnote{In a given population, where succesive infections are possible, lockdown policies aim at diminishing the basic reproduction number $\mathcal{R}_0$, which measure ``expected number of infections generated by a single and (typical) infected individual during their entire infection period'' see Saldan and Velasco for a primer SIR modeling applied to COVID 19}, while managing the extent of the economic losses associated with frozen national and international economies. 

These three examples display cases of management of  spatially distributed renewable resources. Indeed, fish deer populations, and pandemics grow through time, depending on the size of the population. Moreover, they move through oceans, land, jurisdictions and countries. These examples highlight that the management of spatially distributed renewable resources, whether goods or bads, involves at least two layers : managing the stock, and how it moves through space. Indeed, fishing culling, and curing act as stock management measures, while weirs keep the fish in a given area, repellents and fences keep the deers away, and lockdowns avoid spread from infected to non infected people. Finally, in all cases, policies aimed at managing the movement of the resource are more efficient in one way than the other : weirs avoid outflow of fish, but allows inflow; wildlife exclusion fencing often have doors to let animals escape, and to a certain extent, people were prohibited from entering a country more than leaving one during the COVID 19 pandemic. 

% Fencing improves welfare but may not maximize it 

First, the decentralized management of spatially distributed renewable resources is made difficult by the spatial externality they generate. When communities compete for mobile fish, they anticipate part of the school to migrate to other communities, and tend to overharvest, as they do not have secure property right over the whole resource through time \citep{kaffine_unitization_2010}. In the case of deers, free riding on neighbor's culling may deters people to cull the population to efficient levels \citep{costello_private_2017}. In this sense, patch connectivity, in a non cooperative setting, generates inefficiencies. As a consequence, fences appear as welfare improving, as they diminish patch connectivity and therefore contribute to solving the spatial externality. If a fish stock no longer migrates, communities would tend to harvest it in a more sustainable way. If on a given property, deers have no chance of re-entering, then one may undertake efficient culling measures. However, from a welfare perspective, fencing may undo what Nature has rightfully done. Considering spatial heterogeneity in marginal returns to harvesting or culling, and biological productivity, a resource may flow naturally flow to where it is best managed. In this case, although fencing can solve the spatial externality and promote efficient resource use, it would not maximize welfare. Second,  spatially distributed renewable resources live on intricate institutional maps, between private and public land and sea. As a result, optimal harvesting and fencing may be difficult to decentralize. Hence, figuring the second best policy mix to best manage spatially distributed renewables is a challenge. 

This approach can be viewed as application of the spatial trade literature to ecological networks. For example, \cite{donaldson_railroads_2016} shows that railroads have a global effect, as they change the ``market access" of each county, accounting that local changes in ``market access" have spillover effects onto other counties. More generally, to understand the general equilibrium effect of domestic policies on international trade patterns, the use of a structural gravity model is inevitable (e.g. `the new quantitative trade model' e.g. \cite{arkolakis_new_2012}). However, the gravity equation fails at identifying the impact of country specific determinants of trade flows, e.g. multilateral resistance terms \citep{anderson_gravity_2003}. In this article, I analyze the  changes in local fencing patterns have local and spillover effects, and can be seen as changing multilateral resistance terms in an ecological context, and show how they affect each patch, under various management regimes. 

% Case study with bads and deers
% The second best problem in a world with public land and private land
In this article, I focus on the management of ``bads", e.g. species that cause economic damages. This includes rodents, feral pigs, deers, or predators in areas where native species prey are threatened. I develop a theoretical model à la \cite{costello_private_2017}, to understand the interplay between stock and patch connectivity management. Species are harvested, grow and disperse through space, according to immutable environmental factors and expenditures that change connectivity, e.g. fences. Fences have two effects : they keep the bad out (\textit{exclusionary} effect), and they keep the bad in (\textit{trap} effect). In what follows, I assume the exclusionary effect dominates the trap effect. In most cases, exclusionary fencing keeps predators, or damaging species out, while allowing entrapped animals to leave the area\footnote{This can be viewed as an ecological version of inward and outward multilateral resistance terms \citep{anderson_gravity_2003}}..


First, I study the optimal policy mix between stock and dispersal rate management. When costs of control are heterogeneous, the sole owner leverages the spatial arbitrage opportunity, and fences only have an exclusionary effect, the sole owner redirects the population stock to where it controled at the cheapest cost. In doing so, she reduces the population in more expensive patches further than when connectivity is absent. Allowing for resource redispatch, she controls more of the species. When fencing has both an exclusionary and trap effect, cost heterogeneity does not suffice to redirect resource. If biological productivity is larger in relatively costlier patches, trapping them can increase the aggregate cost of the invasive species. Therefore, fencing only occurs when biological productivities and control costs are inversely correlated. 

Second, I characterize the non cooperative equilibrium in harvesting and fencing. When fencing only displays an exclusionary effect, and fencing is costless, every patch owner fences to the maximum. In doing so, they isolate their patch from the rest of the landscape, and control as if they were isolated from other patches. While this results in a more efficient level of control than in the case of uncontrolled spatial dependence, this is not welfare maximizing : as a matter of fact, the non cooperative equilibrium, while solving the spatial externality, does not leverage the spatial arbitrage opportunity provided by heterogenous costs of controling and biological productivities. When fencing displays (unequal) exclusionary and trap effects, best response functions are non monotonous. In this case, increasing fencing is not always optimal, and the Nash equilibrium results in suboptimal fencing, although closer to the optimal solution.

Third, decentralizing the optimal policy on public and private land may prove impossible. Therefore, I investigate the second best allocation, where some patches of land can enforce the optimal policy mix, while others cannot, and fencing is restricted. I generalize insights from \cite{costello_optimal_2008} and \cite{costello_private_2017} to understand how to optimally control the stock and connectivity when only a subset of patches can be regulated. I show that implementing the first best policy mix, which reshuffles resources to the most cost effective patch is always best. The second allocation is decentralizing an uncoordinated equilibrium, as the spatial externality is resolved and future damages and costs are internalized. However, when a policy maker can only choose 1 instrument, decentralizing optimal fencing with uncoordinated control is the worst outcome, while decentralizing optimal control with uncoordinated fencing is not the worst outcome. Finally, I use a simplified empirical application, using simulated data, to illustrate the optimal control and fencing in the presence of cost and biological heterogeneity, as well as the non cooperative equilibrium. Additionally, I characterize the welfare effects of different management scenarios, depending on the starting policy ground. 

This article proceeds as follows : section \ref{sec:related_literature} draws lessons from the existing literature, section \ref{sec:model} explains the models main mechanisms. In section \ref{sec:optimal_management}, I establish results for the optimal fencing and controling of a public bad, while section \ref{sec:decentralized} looks at the uncoordinated equilibrium. Eventually, section \ref{sec:numerical} illustrates the findinds, section \ref{sec:conclusion} concludes. Proofs can be found in the appendix (see section \ref{sec:appendix}).

\section{Related literature}
There is a vast literature that investigates the optimal control, eradication and detection of invasive species (see Epanchin Niell for a review). A much scarcer one looks at the spatial nature of the management of public bads and/or invasive species. Early approaches,\cite{huffaker_optimal_1992},\cite{bhat_controlling_1996} analyze various management regimes (cooperative, isolated, and coordinated) to deal with the presence of beavers on private land. Movement between patches corresponds to a density dependent pattern, which is, funny enough, an adaptation of Stenseth's ``\textit{social fence}'' hypothesis \cite{stenseth_social_1988}. In this framework, migration is entirely driven by relative densities. Therefore, optimal stock management needs to account for these migratory effects. With this analysis, \cite{huffaker_optimal_1992} and \cite{bhat_controlling_1996} limit themselves to two patches, for analytical and computational tractability. A different approach, viewing space as a continuum, has considered options to halt the progression of an invasive species, using barreer zones, to ultimately slow the rate of spread \cite{sharov_bioeconomics_1998}. While theoretically appealing, this approach may not be suited for operational concerns, whereby optimization on a continuum space is difficult, especially in various directions. In the wake of \cite{brown_metapopulation_1997}, \cite{bulte_metapopulation_1999}, numerous models of invasive species have been developed in economics, taking advantage of familiar optimization structures. For example, \cite{blackwood_cost-effective_2010} develop a linear quadratic framework to study the control of an invasive plant species. Taking advantage of the stock independent nature of migration patterns and of the linear quadratic structure, the authors solve the control and prevention problem at a large spatial scale. In more recent work, \cite{costello_private_2017} develop a large scale model of public bads, characterized by exogenous dispersal, and analyze the potential for eradication in a connected landscape. In doing so, they analyze the effects of varying connectivity parameters, without acknowledging for the potentially endogenous nature of dispersal. Finally, a wealth of papers, in the wake of \cite{sanchirico_bioeconomics_1999}, several papers \citep{albers_invasive_2010, ambec_regulation_2012} have investigated the use of policies to halt the spread of invasive species, including mandatory refuges, albeit uniform. While these articles view dispersal as a characteristic that can be influenced, they do not consider the optimal management, or lack thereof, of dispersal. Finally, \cite{janmaat_sharing_2005} highlights the role of dispersal in a fishery, and other parameters, to assess the extent of the tragedy of the commons. Interestingly, in that article, Janmaat states that `` \textit{until ‘fences’ are available to contain the ‘wandering’ offspring, management zones would have to be large. This would minimize the spillover, bringing the incentives of the ‘owner’ into line with maximizing the total returngenerated by the resource}". The contribution of the article is framed as how to adapt regulation to a given migration pattern. In this article, I reverse the approach for terrestrial species, of sufficient size such that their dispersal can (more or less) be managed. In this article, I build on these frameworks by using a discretized, raster-type landscape, with metapopulation dispersal across patches. Instead of analyzing how policies should adapt to dispersal, and I analyze how policies can shape dispersal, and what happens in the case where management is incomplete.

\label{sec:related_literature}
%\begin{itemize}
%\item Burnett, 2008
%\item Costello and Quérou 2017
%\item Bhat and Huffaker 1996 - Controlling transboundary wildlife damage: modeling under alternative management scenarios

%\item Epanchin Niell \& Wilen 
%\item Blackwood et al 2010
%\item Broadly speaking, the literature on network games provides interesting insights highlighting that players’ behaviors
%are influenced by those around them (Jackson and Zenou, 2014)
%\item Olson and Roy, 2002
%\item Janmaat + check de la base
%\item Managing Urban Deer, Rondeau
%\item Gender-Based Harvesting in Wildlife Disease Management
%\item Jointly determined ecological-economic tradeoffs in wildlife disease management 
%\item Optimal harvesting of a plant-herbivore system : lichen and reindeer in Finland
%\item A note on the economics of bioinvasions
%\item An age-structured bio-economic model of invasive species management: insights and strategies for optimal control
%\item Invasive species management in a spatially heterogeneous world : effects of uniform policies
%\item Spatial management of Wildlife Disease
%\item Bioeconomics of managing the spread of exotic pest species with barrier zones
%\item Spatial Management of Invasive Species: Pathways and Policy Options
%\item Optimizing spatial and dynamic population based  control strategies for invading forest pests
%\item On Prevention and Control of an Uncertain Biological Invasion
%\item THE ECONOMICS OF CONTROLLING A STOCHASTIC BIOLOGICAL INVASION
%\item Metapopulation dynamics and stochastic bioeconomic modeling
%\item The economics of managing infectious wildlife disease 
%\item Bioeconomic management of invasive vector-borne diseases
%\item OPTIMAL TRAPPING STRATEGIES FOR DIFFUSING NUISANCE-BEAVER POPULATIONS
%\item Optimizing the Use of Barrier Zones to Slow the Spread of Gypsy Moth (Lepidoptera: Lymantriidae) in North America
%\item Application of distributed parameter control in wildlife damage management
%\item Optimal Control of Vaccine Distribution in a Rabies metapopulation model
%\item Pest as a common property resource : a case study of alfalfa weevil control
%\item Pests: Sustained Harvest versus Eradication
%\item Spatial Dynamics of Optimal Management in Bioeconomic Systems
%\end{itemize}


%Coordination of action through stock management but also maybe with fences etc. 

\section{A dynamic spatial model of renewable bads management : fencing and controlling}
\label{sec:model}

This model is adapted from \cite{costello_private_2017}. It conserves the main features and includes an endogenous determination of landscape connectivity.

\subsection{Spatial ecology}
Assume $N$ patches indexed $i\in \{1, ..., n\}$ with a renewable resource. In a given period, the resource stock $X_{it}$ is harvested by $h_{it}$, and grows according to the remaining stock, defined as $e_{it} = X_{it} - h_{it}$, such that the pre-migration population in patch $i$ in $t+1$ is $g_i(e_{it})$ such that $g_i'(e_{it})\geq0, g_i''(e_{it}) \leq 0$.

Moreover, after the resource grows, it disperses through space (see fig. \ref{fig:timing} for a summary of the model timing). This is consistent with continuous metapopulation models \citep{sanchirico_bioeconomics_1999, bulte_metapopulation_1999}, although discretized \citep{costello_private_2017}. I assume that dispersal exclusively depends on exogenous, immutable environmental characteristics, and fencing. Density effects on migration rates are not considered in this model.

Dispersal rates between patches depends on directional fencing expenditures in both patches, with $d_{ijt+1} \equiv d_{ijt+1}(f_{it}^j, f_{jt}^i)$, where $f_{it}^j$ measures the amount of fencing in patch $i$ in direction of patch $j$. The inflow of invasive species from $i$ to $j$, $d_{ijt+1}(f_{it}^j,f_{jt}^i)$ decreases with $f_{jt}^i$. I call this the ``exclusionary effect'': fences keep nuisances out of $j$. When fencing in $i$ at $f_{it}^j$, the outflow of invasive species from $i$ to $j$ decreases as well, as species get trapped in $i$. This effect is the ``trap effect'' : fences trap the nuisance in. In real life applications, either the exclusionary or trap effects tend to dominate. Indeed, actual traps have a limited exclusionary effect, while exclusionary fencing always features a potential escape for trapped animals. Nonetheless, I focus on a symmetric case : in this set-up, fences keep as much in as they keep out.Fencing reduces the inflow from $i$ to $j$ at a decreasing rate, whether it is undertaken in patch $i$ or $j$. The rate of patch retention $d_{iit+1}$ is the remainder after migrations from $i$ to $j$.
Dispersal rates are ultimately affected by immutable environmental factors (landscape discontinuities such as roads, rivers, moutains; altitude and terrain ruggedness etc). These immutable factors are pairwise symmetric. Eventually, dispersal rates sum to $1$. Therefore : 
\begin{align}
d_{ijt+1} : \mathbb{R}^+ \times \mathbb{R}^+ \to [n_{ij},m_{ij}] \subset [0,1] \\
\underbrace{\frac{\partial d_{jit+1}}{\partial f_{jt}^i}}_{\text{Exclusionary effect}}=\underbrace{\frac{\partial d_{ijt+1}}{\partial f_{it}^j}}_{\text{Trap effect}} \leq 0\\
\sum_{j\neq i}^N d_{ijt+1}(f_{it}^j, f_{jt}^i) + d_{iit+1} =1
\end{align}

Where $n_{ij}, m_{ij}$ are the immutable bounds to dispersal rates, and second-order derivatives are (weakly) positive. 

%Finally, connected patches can be seen as a graph with vertices $i \in \{1, ..., n\}$ and directional, weighted edges $d_{ijt+1}$. Hence, dispersal at the landscape scale can be apprehended using an $n \times n$ matrix $\mathbf{D}_{t+1}$ such that $\mathbf{D}_{t+1}(i,j) = d_{ij+1}(f_{it}^j, f_{jt}^i)$. Each row $i$ of $\mathbf{D}_{t+1}$ represents outflow from $i$ to all patches (including self retention in $i$) and each column $j$ represents the inflow from all patches to $j$ (including self retention in $j$). 

\subsection{Spatial economy}

The presence of bads is costly in each patch via two channels, modeled as in \cite{costello_private_2017}. First, the presence of bads implies property specific control expenditures. The larger the stock, the lower the marginal cost of control, hence accounting for a stock effect, where the marginal cost of control $c_i(s)$ is decreasing with stock size, $c'_i(s)<0$. The total cost of controlling down to residual stock $e_{it}$ is $\int_{e_{it}}^{X_{it}}c_i(s)ds$. 

Additionally, the presence of the residual stock causes heterogeneous marginal damages (for example, deers cause more damages to orchards and managed forests  than to meadows) $k_i(s)$, which increase with stock size $k'_i(s)>0$, resulting in convex damages. The total damages caused by the residual stock is $\int_{0}^{e_{it}}k_i(s)ds$.

Eventually, fencing is costly, with heterogeneous costs (driven by terrain, difficulty of access, type of fence etc) across patches. The marginal cost of fencing $\gamma_i(s)$ is weakly increasing with fencing $\gamma_i'(s)\geq 0$ and $\gamma_i(0)>0$. The total cost of fencing is  $ \sum_{j\neq i} \int_{0}^{f_{it}^j} \gamma_i^j(s)$.

The total cost in each patch $i$ and period $t$ is : 

\begin{equation}
C_i(e_{it}, X_{it}, f_{it}^1, ..., f_{it}^j) = \int_{e_{it}}^{X_{it}}c_i(s)ds + \int_{0}^{e_{it}}k_i(s)ds + \sum_{j\neq i } \int_{0}^{f_{it}^j} \gamma_i^j(s)
\end{equation}

The patch-period specific cost depends on current patch specific decisions, as well as past decisions by other agents, which influence the stock of bad in patch $i$ at the beginning of period $t$. 

\section{The decentralized equilibrium}

I assume that the $N$ patch owners can decide individually their fencing level, as well as their control level. The decision is complicated because each player needs to know how the initial stock level across the whole landscape, as well as the distribution of marginal cost and damages, growth and fencing costs. Moreover, it has to anticipate how these decisions will affect the future : a lower level of control today will lower the future marginal cost of control, but may cause larger damages.

Mention the equilibrium concept?

Finally, the strategy space is two-dimensional : indeed, each player not only has to chose their control level (as in \cite{costello_private_2017}) but also their fencing level. 


To simplify the decision, I develop a two-stage dynamic game, according to Figure \ref{fig:timing}. In each period, patch owners first decide on the fencing level in each direction, conditional on fencing decisions in other patches. After having observed the fencing levels in every direction, and how dispersal is affected, they decide on the control level in their patch. This game is solved using backwards induction. Best response functions for controls are computed, conditional on fencing levels. Because each patch owner's decision depends on the control level, a subgame non cooperative equilibrium emerges where equilibrium controls only depend on fencing levels, and patch specific characteristics $\hat{e}_{it}(\mathbf{f})$ as in \citep{costello_private_2017}. Taking this equilibrium result, patch owners choose the level of fencing to minimize their total cost.






\begin{itemize}
\item First, establish the second step equilibrium
\item Then Establish the first stage equilibrium
\item Should I enunciate the FOCs first?
\end{itemize}


\newpage



