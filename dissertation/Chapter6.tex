\phantomsection
\addcontentsline{toc}{chapter}{\textbf{Conclusion}}
\chapter*{Conclusion}
\onehalfspacing

This dissertation explores the bioeconomic modeling of biodiversity loss, addressing two key questions: (1) How do endogenous spatial processes impact the drivers of biodiversity decline, and how can they be managed to mitigate this decline? (2) What role does strategic behavior play in exacerbating or alleviating these drivers, and how can bioeconomic models account for these behaviors to inform effective conservation policies?

By combining insights from both economic theory and ecology, this dissertation contributes to the vast body of literature that seeks to develop integrated approaches to biodiversity, bioeconomic modeling. Through a combination of spatial models, dynamic optimization, and strategic behavior analysis, the research presented here offers both theoretical and practical insights into how biodiversity can be conserved in a complex and interconnected world.

\phantomsection
\addcontentsline{toc}{subsection}{Managing Endogenous Spatial Processes}
\subsection*{Managing endogenous spatial processes}

In chapter 2, the research focuses on the tension between managing space for conflicting objectives, particularly the trade-off between reducing wildfire risk and conserving biodiversity habitat. Forest patches display dynamic successional stages - each forest patch grows through time - , contributing to both fire risk and biodiversity, depending on their spatial arrangement. Their absolute and relative location in space, e.g. own characteristics and the degree to which they are connected to neighboring patches matter for connectivity. The study reveals that managing habitat and wildfire risk within landscapes is inherently complex due to the non-convexity of the underlying connectivity : the successional stage and location relative to neighbors crucially matters. I demonstrate that under certain conditions, it is possible to maintain habitat connectivity while limiting wildfire risk, but this balance becomes precarious in the face of climate change, or with limited policy budgets. 

This chapter confirms results from existing studies in terms of the production possibility frontier between habitat and wildfire risk \citep{calkin_modeling_2005}, the decreasing marginal efficiency of treatments \citep{wei_optimization_2008, yemshanov_exploring_2022} and provides an interpretative framework using graph theory for existing results \citep{minas_spatial_2014,
rachmawati_optimisation_2016,
konoshima_spatial-endogenous_2008, yemshanov_detecting_2021}. 

This contribution is significant in that it highlights how spatially dynamic features - specifically, the centrality of certain patches within a network - affect the efficacy of conservation efforts. Managing central nodes, which have outsized impacts on the connectivity of the landscape, proves to be a critical factor in reducing risk when habitat connectivity is not a policy priority. However, when it is, treatments should focus on less central patches, and their number should gradually dimnish. Graph theory provides an effective framework to understand the location of treatments at a small scale. This chapter also provides a methodological insights relative to the scalability of small scale results. In small graphs, connectivity metrics tend to be very correlated. Hence, information from small lansdcapes provides limited information for large scale policy design. 

However, the research also demonstrates that the non-convex nature of spatial connectivity of habitat and wildfire patches  complicates optimization. Traditional methods, particularly those that rely on linear or convex assumptions, continuity, or limited numbers of variables such as dynamic programming, fail to provide optimal solutions in the high-dimensional, discrete space of ecological management. In this context, the dissertation introduces the use of heuristics and approximations to manage complexity, though at the cost of reducing the planning horizon and the analytical tractability of the model.\\

Chapter 3 extends this analysis to the management of invasive species, where the strategic deployment of ecological fences is examined along with \textit{in situ} control policies. The chapter revisits a model of mobile public bads \citep{costello_private_2017}, and makes connectivity both a tool and a challenge for managing pests. The findings show that optimal connectivity can be designed to contain invasive species effectively : redirecting the flow of pest species to where it is best controlled, or least invasive is welfare enhancing. This changes the results of previous models \citep{costello_private_2017} that treated connectivity as exogenous to the system. By treating connectivity as a decision variable, this research provides new insights into how species can be controlled more efficiently through spatial interventions. However, the decentralized management of connectivity complicates matters: in some cases, suppressing a spatial externality, such as species spread, can leave spatial arbitrage opportunities untapped, where heterogeneity across the landscape could have been leveraged for better ecological and economic outcomes.

Both chapters emphasize the importance of considering spatial dynamics and connectivity in conservation strategies.  However, these chapters highlight an incompatibility frontier for current bioeconomic models \citep{levin_social-ecological_2013}. Increasing the number of state variables (e.g. considering more interrelated spatial units), incorporating temporal dynamics (increasing the planning horizon), and representing complex ecological interactions (non-convexities in aggregate landscape connectivity, decisions which depend on the state variables) lead to complex problems at the frontier of research. In this case, analytical tractability is difficult to maintain. Hence, including these layers of mathematical complexity changes the purpose of models: while many models were first and foremost heuristic, building fictitious worlds to glean insights on specific policy issues, they become more prescriptive and predictive  \citep{varenne_epistemologie_2014} and their heuristic role is challenged. 


\phantomsection
\addcontentsline{toc}{subsection}{The role of strategic behavior in biodiversity management}
\subsection*{The role of strategic behavior in biodiversity management}

The second major theme of this dissertation is the impact of strategic behavior on biodiversity management, explored in Chapters 3 and 4. In Chapter 3, the strategic behavior of landowners in the management of connectivity and invasive species is examined. The findings reveal that when landowners act in their own interest - overfencing their properties to prevent the spread of invasive species - spatial fragmentation results. This defensive strategy tends to internalize the damages of invasive species, but prevents the full exploitation of spatial heterogeneity in the landscape, where different patches could serve as containment zones for invasive species at lower management costs. The chapter illustrates that when actors fail to consider the broader ecological benefits of coordinated action, they miss opportunities to improve both ecological and economic outcomes. Overfencing not only leads to inefficient use of resources but also hinders the natural movement of species that could have been managed more effectively through strategic, coordinated interventions.

In Chapter 4, the dissertation turns to the strategic behavior of actors within market systems, focusing on the case of the \textit{Totoaba macdonaldi}, an endangered species targeted by poaching. The chapter reveals that vertical monopolies do not necessarily lead to overharvesting or resource depletion. Instead, the relationship between upstream and downstream actors - particularly the constraints imposed by downstream producers - can limit the extent of overexploitation. As fishing becomes more costly when stocks decrease, this puts a bound on the capacity of the vertical monopoly to supply large quantities while maintaining its margin, on top of classical demand related constraints related to its elasticity. Moreover, Bertrand competition, often expected to lead to aggressive overharvesting \citep{bulte_economic_2005, damania_economics_2007}, is tempered by these same constraints. Contrary to existing results, we show that Bertrand competition does not necessarily lead to market flooding and stock collapse.  The introduction of conservation aquaculture offers a viable solution to poaching, provided that farming costs remain low and poaching penalties are enforced. In a second best world, where a fishing ban is difficult to enforce, and property rights are difficult to assign, leveraging \textit{de facto} property rights and designing smart market conditions may provide a better alternative than increasing law enforcement. 

This analysis underscores the importance of considering strategic interactions in biodiversity management, whether in spatially explicit contexts, as in Chapter 3, or within market structures, as in Chapter 4. In both cases, strategic behavior complicates the optimization of conservation outcomes but also offers opportunities for leveraging market or spatial incentives. By recognizing and incorporating strategic behavior into bioeconomic models, policymakers can design interventions that account for the realities and consequences of human behavior in both conservation and market systems.


\phantomsection
\addcontentsline{toc}{subsection}{Policy implications}
\subsection*{Policy implications}

The findings of this dissertation suggest several important policy recommendations. First, for wildfire and habitat management, spatial optimization can guide treatment operation, and the dynamic appraisal of landscapes should be in the landplanner's toolbox.
Allocating resources to central nodes - those patches with the greatest influence on landscape connectivity - will help maximize the benefits of fuel treatment interventions. However, when biodiversity habitat is factored in, the careful design of biodiversity corridors is key, and graph theoretical procedures can help. 
Although our results suggest that current network centrality measures fail to be leveraged on large scales, research development in centrality metrics can considerably increase the efficiency of multi-objective fuel treatments.

Second, in the management of mobile public bads \citep{costello_private_2017}, policymakers should focus on coordinated connectivity control rather than allowing decentralized actors to overfence their regions. Policies should focus on optimal ecological network design, where temporary fencing and containment policies can achieve larger welfare improvements than control alone. There is no one-size-fits-all policy recommendation when heterogeneity is factored in. Nonetheless, information on the distribution of costs and biological productivities should guide optimal fencing policies: if large costs zones have low biological productivity or stock, they should be isolated, to avoid either contaminating other patches or receiving inward dispersal, which comes at substantial costs. To conclude, this article goes in the direction of spatially explicit, incentive compatible policies to form coalitions for service provision such as agglomeration bonuses \citep{parkhurst2002agglomeration, bareille_agglomeration_2023}. 

Third, in the context of conservation farming and poaching, the case of \textit{Totoaba macdonaldi} suggests that market-based solutions can play a pivotal role in reducing pressures on wild populations. Subsidies for aquaculture, combined with demand reduction campaigns, and sustained law enforcement against poaching, can help tip the balance toward sustainable conservation outcomes. In settings where property rights are difficult to assign and regulate, and local law enforcement is difficult, the use of command and control approaches like trade bans under CITES may not be efficient. Using market based instruments such as trade ban exemptions can be a tool to curb poaching. 
Policymakers must also consider the market structure when designing interventions, as both monopolistic and competitive dynamics affect the incentives for conservation. Policy design at the collective level can use insights from imperfectly competitive market structures such as totoaba's. As a matter of fact, these structures can be welfare enhancing compared to \textit{status quo}, result in decreased fishing effort, increased biomass and increased (tax) revenues \citep{englander_fish_2023}.

\phantomsection
\addcontentsline{toc}{subsection}{Limitations and future research}
\subsection*{Limitations and future research}

While this dissertation provides valuable insights, it also faces several limitations.\\
In chapter 2, I use a bounded dynamic vegetation model to maintain the possibility to analyze integrally the set of initial conditions. In doing so, I restrict the potential states of the world and the depiction of the relationship between wildfire risk, habitat suitability, and successional stage. Therefore, increasing the temporal depth of the model is as much of concern as increasing the scale, to allow for different fire-habitat-successional stage relationships and better guide policy.
Additionally, including economic heterogeneity in the costs of treatment and potential damage would benefit policy making. Finally, centrality measures are shown  not to scale well on larger graphs. We believe insights from the small scale results are valid, but the relevant large scale centrality measures, and additional computational experiments, are required to sustain that claim. Therefore, avenues for future research involve different approaches to build robust information about the temporal and spatial location of treatments.  I plan on sampling medium to large scale location and solve the repeated and dynamic optimization procedures with genetic algorithms. Second, using the results from this first step optimization procedure, I plan on characterizing the solution using graph theoretical networks and training neural networks to recognize optimal treatment patterns, and help guide larger scale optimization procedures. 

In chapter 3, I so far restricted the analysis to 2 players, and the endogenous formation of a 2 node graph. Further analysis implies increasing the number of players to really study the emergence of complex network structures, where the structuring into components (e.g. disconnected subgraphs) may emerge as optimal policy options, diplaying positive connectivity within component. In doing so, I would like to study the properties of efficient policies, how they shape ecological-economic network, and how graph theoretical measures can help dealing with complex optimization problems. 

In chapter 4, our policy recommendations build on uncertain economic and ecological data, as ecological data is expensive to get, and illegal market data difficult to encounter. Data scarcity on ecological processes is a key feature of bioeconomic modeling. In my future research, I want to integrate more data sources from ecology, especially at a fine spatial resolution, including products from satellite imagery and Geographic Information Systems (GIS). 
With these limitations in mind, our policy recommendations should be taken cautiously.  \\
Additionally, several additional layers could be included. First, we do not acknowledge for dynamic pricing interactions, where the stockpiling of swim-bladders is an option to increase prices through time, as we expect on-the-ground seizures to remain a threat. However, these effects have been shown to matter \citep{Kremer2000}, and further analysis is required. Second, we restrict our analysis to specific fish behavior, where fish do not have specific migratory routes or spawning grouds, and thus, the catch decreases proportionally to population size. One fruitful research avenue is to enhance the set of fish school behaviors considered in the model, such as stock hyperstability (i.e. when costs of fishing do not increase as the population decreases, because fish tend to group at specific locations, for mating processes for example), to test the validity of our policy recommendations. 


Future research avenues are both methodological and thematic. In order to grasp the intricacies of the drivers of biodiversity loss drivers, I want to study how to increase the total complexity of models, by refining the aspects I traded for others among chapters. \\

Increasing the spatial resolution of models hinders the use of traditional dynamic optimization techniques, such as dynamic programming \citep{Bellman}. While techniques are being developed to increase the spatial resolution of models and limit their computational burden \citep{brumm_adaptive_2017}, these techniques may not be sufficient. Systems of ordinary differential (or difference) equations can be more convenient to solve, but partial differential (or difference) equations \citep{brock_pattern_2010, brock_2020} systems allow spatial variables to be considered in a continuous way (e.g. 1, 2 or 3 dimensions) and can be of interest to study large spatial issues. 

The non-convexities identified in the spatial models, particularly in Chapter 2 (and likely present in the development of chapter 3 as network size is increased) present significant challenges for scaling these models to real-world applications. In my future research, I  want to explore ways to develop scalable optimization methods that can handle the complexity of spatial dynamics without sacrificing computational efficiency. Machine learning offers a promising avenue for addressing these challenges by enabling the development of more flexible, high-dimensional models, such as statistical learning and scaling up from heuristic optimization methods applied to medium size landscapes. 

Another limitation is the lack of stochasticity in the models presented. Real-world ecological processes are subject to a wide range of uncertainties - random species invasions, wildfire ignitions, and market fluctuations, among others. Incorporating stochastic elements into the models would provide a more robust foundation for making policy recommendations that can withstand the unpredictability of ecological and economic systems. However, as stated earlier, this level of complexity triggers operational research problems, where the solvability of such models is difficult. The climate macroeconomics literature has developed tools to incorporate large spatial scale modeling with stochastic processes into optimization frameworks \citep{cai_modeling_2020,fernandez-villaverde_climate_2024}, thus paving the way for similar approaches for biodiversity economics. 

As biodiversity decline is a multispecies and multiscale phenomenon, future research should expand beyond the single-species models used here to consider multi-species interactions and community-level approaches. Conservation efforts should focus not only on minimizing economic "bads," such as invasive species, but also on optimizing the broader ecological landscape, including the provision of Nature's Contributions to People (NCPs). Indeed, the management of connectivity should not only be apprehended through the hardships of mobile public bads, or single species habitat suitability but also through the lense of other biodiversity contributions. For example, in the case of species providing positive NCPs, ecological connectivity is of great value, as it provides insurance against stochastic population variation \citep{loreau_biodiversity_2003}. While habitat-based models provide a basis for such endeavor, the analysis of population dynamic for species with different habitat requirements raises issues regarding the choice of spatial scale to aggregate different species, and compare patches.

In this dissertation, I have shown that bioeconomic models serve as a platform for the analysis of biodiversity through its ecological and economic lense, in a trully interdisciplinary fashion. Gradually, they have been refined to encompass advances from both economics and ecology. As they are models with a mathematical formulation, their resolution is made difficult with increasing complexity, and analytical tractability vanishes. With data being scarce, they can be difficult to calibrate, but increases in both ecological and economic dataset availability at a fine temporal and spatial resolution are changing calibration issues. I firmly believe that this method is suited for the challenges of the future. Data driven methods are perfect for analyzing the past, but may fail to predict the future. In the context of the ecological crisis, developing models to understand, predict, and guide policy making is paramount. As I want to continue studying landscape connectivity and its value, other approaches are complimentary. This thesis cruelly lacks original empirical evidence on the value of connectivity. This strand of my research is burgeoning, and modern causal inference methods can be used to analyze changes in the value of connectivity, as they have been recently used to understand the value of individual species \citep{frank_economic_2024, frank_social_nodate}. The literature on the estimation of spillover effects of policy in environmental and epidemiological systems is a lively field and has provided great methodological and empirical constributions \citep{deschenes_quasi-experimental_2018,reich_review_2021}. The literature focuses on the causal estimation of treatment effects with known (or assumed dispersal) patterns. Chapters 2 and 3 highlight that shocks to parameters such as growth, economic costs and damages can have far reaching consequences in terms of network structure : shocks not only have impacts on local populations, but also on the structure of the network and the pattern of spillovers. Recent advances in econometrics \citep{comola_treatment_2021} foster new ways to understand changes relative to landscape connectivity with endogenous changes in spillovers. \\



\clearpage
%% Conceptual challenges
{\footnotesize
\bibliographystyle{abbrvnat}
\bibliography{bibliography_these}
}