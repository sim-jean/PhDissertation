\chapter*{Introduction}
\label{Introduction}

What should I say in the general introduction of my thesis? 

\begin{itemize}
\item Une page qui résume : le déclin de la biodiversité, l'économie bioéconomique, la contribution méthodologique de la thèse, et les résultats nouveaux. 

\item Introduction générérale thématique : le déclin de la biodiversité / 3-5 pages avec des graphiques. 
\begin{itemize}
\item Le déclin de la biodiversité de façon générale
\item Par ecosystème et taxon : on parlera alors de poissons, d'espèces invasives, de forêts
\item En rajoutant un petit quelquechose sur le changement climatique
\item Les causes sont à chercher du coté des hommes
\end{itemize}

\item D'où la necessité d'une approche par les sciences sociales, et l'économie peut y apporter beaucoup, elle l'a déja fait : social ecological systems, JEEM. 
\item Pourquoi l'économie et la modélisation bioéconomique?
\begin{itemize}
\item Un problème d'externalité global, de biens publics, de valeurs d'options, d'informations e.g. tous les problèmes spécifiques à l'économie de l'environnement
\item Un outil permettant la modélisation et la prospective: c'est à dire la description, sur base axiomatique, des comportements, à la fois individuels, non coopératifs, et de politique publique. 
\item Qui permet d'articuler grâce à un langage partagé différent champs de connaissance, notamment de dialoguer avec les sciences du vivant
\item Nonobstant les contributions
\item Il existe des champs d'application inexplorés, ou des questions importantes non résolues : bien lister ici les dimensions : marché, espace, politique publique/gestion privée. \\
From the review:
\begin{itemize}
\item Stochasticity : how stochasticity governs ecological and economic processes is a recent strand, but studied (costello, augereau véron, quaas and Baumgartner). Remain an open question as to how do people respond to uncertain and risky contexts, with the limited possibility of insurance; mentionner des choses à la Romain, genre risques reliés : stochasticité de la ressource écologique et incertitude autour de la détection des invasive species etc. 
\item Community perspective for ecosystem based conservation planning, e.g. interaction between several species at the landscape level
\item The role of spatial processes, for movement, different habitat quality, understanding how space changes results in combination with other determinants of bioeconomic models (Sanchirico, Costello as well)
\item Intricate property rights, ranging from the most localy complex in terms of types of interactions (externality locally), local competition for resource, different levels of political management, but also different types of market structure

\item Incorporating different value types, including indigenous knowledge, in bioeconomic modeling for policy making

\item incorporating more evidence from quasi experimental methods in the results
\end{itemize}
\end{itemize}

\item \textbf{ On a donc deux éléments bien identifiés dans la recherche et les questions qu'ils posent }: l'espace et la structure de marché. 
\item La structure de marché : 
\begin{itemize}
\item Question bien vieille dans la litérature sur les ressources naturelles: Solow, Hotelling etc
\item Moins bien tranchée sur la question des ressources renouvelables : rhinos (AER), foresterie etc ... 
\item On l'étudie donc dans un cas précis, c'est le chapitre 2
\end{itemize}
\item La question de l'espace : 
\begin{itemize}
\item L'espace est une dimension importante à prendre en compte, car il conditionne l'exploitation des ressources, autant que comment les décisions doivent être prises : ce qui change (Sanchirico et Costello)
\item C'est un challenge qui pose des questions de politique publique : la gestion de l'espace, dans un contexte de fragmentation, de prévention des risques etc est cruciale 
\item Il faut aussi comprendre comment l'espace, et les processus écologiques qu'il construit, sont formés par les individus [à raffiner] c'est le chapitre 3
\end{itemize}
\item Bien mentionner que l'intégralité des données, codes etc sont disponibles gratuitement et librement. 
\end{itemize}

\textbf{What's left to do}
\begin{itemize}
\item List contributions from the literature review, and find a way to put more into it. 
\item Find references and graphs from the institutions to document biodiversity loss
-> IPBES?
\end{itemize}
\clearpage
\section{New intro}

\begin{enumerate}
\item Lay the facts : global biological diversity decline
\begin{itemize}
\item What is biological diversity? How can we define it? What values does it call for? \textit{Est ce que ça a vraiment sa place ici?} - peut être plutôt note de bas de page, sur les difficultés du concept etc, mais on se dit richness and abundance across the world
\item An overall decline over time : 
\begin{itemize}
\item Across taxa
\item Across ecosystems
\item Across types of species
\end{itemize}

\item The causes are man made, on land and at sea : overexploitation, habitat destruction and fragmentation
\\
Rajouter une bonne couche sur la nécessité de (i) pas trop exploiter et pourquoi; et (ii) sur la fragmentation de l'habitat (avec pas mal de concepts écologiques, genre surface, fragmentation, stepping stones etc)
\end{itemize}

\item Need a unifying method, conceptual body, to remedy this crisis : find pathways towards a sustainable future etc\\
$\Rightarrow$ Bioeconomic modeling : what it is

\begin{itemize}
\item What it does : mix together ecological dynamics and economic drivers
\item Strengths : process based, out of sample, policy design
\item Applied to various ecosystems : forestland a lot, oceans and fisheries, and agricultural land to have several \\
Maybe list some results?
\end{itemize}

\item The current state of biodiversity calls for an ``ecosystem centered'' management\\
\textit{Unconfortable with that, but it allows to introduce bioecon modeling, and highlight the shortcomings based on empirical facts}
\begin{itemize}
\item Ecosystems feature different uses, different species, offer different risk and benefits etc : cases of forests (risks of wildfire, habitat to biodiversity, forestry industry, leasure); and oceans (fisheries, conservation of habitat, leasure).
\item The literature has gradually evolved towards encompassing the many dimensions of ecosystem based management, but shortcomings remain : 
\begin{itemize}
\item Stochasticity : how stochasticity governs ecological and economic processes is a recent strand, but studied (costello, augereau véron, quaas and Baumgartner). Remain an open question as to how do people respond to uncertain and risky contexts, with the limited possibility of insurance; mentionner des choses à la Romain, genre risques reliés : stochasticité de la ressource écologique et incertitude autour de la détection des invasive species etc. 
\item Community perspective for ecosystem based conservation planning, e.g. interaction between several species at the landscape level
\item The role of spatial processes, for movement, different habitat quality, understanding how space changes results in combination with other determinants of bioeconomic models (Sanchirico, Costello as well)
\item Intricate property rights, ranging from the most localy complex in terms of types of interactions (externality locally), local competition for resource, different levels of political management, but also different types of market structure

\item Incorporating different value types, including indigenous knowledge, in bioeconomic modeling for policy making

\item incorporating more evidence from quasi experimental methods in the results
\end{itemize}
\end{itemize}
\item In this dissertation, I chose to focus on space and market structure across 2 different types of ecosystems
\begin{itemize}
\item Space \& Market structure : make up policy, conceptual and methodological challenges, as they adress the principal causes of global biodiversity decline and foster technical and methodological challenges, as highlighted by our review and other. 
\item Space on terrestrial landscapes is important to tackle main causes of species decline: 
\begin{itemize}
\item Classical arguments : featuring space in traditional models allows to understand better aggregate behavior (Sanchirico); externalities arise from spatial dependence, and optimal behavior can be characterized in some cases (Costello)

\item Forests support a variety of functions and risks, benefits and costs; these risks and benefits have distinct spatial patterns, and occur at a large scale, calling for an aggregate scale approach; different phenomena are characterized by different scales, different time frames and different ??? but connectivity is an important feature that structures ecological processes; raises specific challenges with spatial optimization; spatial management of connectivity with multiple use/functions ecosystems is complicated, but useful in terms of policy and methods\\
Additionally, in a context of resource scarcity, space can be a driver of policy success for both resource preservation and wildfire prevention. 
\item Spatial heterogeneity drives movement; spatial ressources are difficult to manage as they create spatial externalities, that are dynamic through time; it has long been considered that movement is an exogenous process, or driven exclusively by ecological features; besides fragmentation, as well as corridors etc, human actions at a more local scale can change the pattern of spatial movement, especially in the case of invasive species. Petite envolée sur le fait que la géographie est le résultat d'un processus social? When possible, people may fence, to resolve the externality; in doing so, they may have a better use of the resource (e.g. halt overexploitation) but undermine landscape connectivity (e.g. increase fragmentation); it is shown that not all fences are bad, and connectivity is good if heterogeneity exists; 
\end{itemize}
\item Market structure matters for resource exploitation
\begin{itemize}
\item In the harvesting literature, it is often considered that marginal revenue is constant as prices are determined by an international market. Additionally, open access is a key feature of the commons, and a small group of people seldom controls a complete market.
\item However, some specific cases of endemic resources, such as wildlife trade, in informal contexts, can be characterized by restricted access
\item A monopolist may be a conservationist's best friend : intuitively, restrict quantities to maximize profits. This is dependent on market specifics, discounting etc (Hanesson)
\item Additionally, competition in intricate settings can trigger weird dynamic, according to existing papers 
\end{itemize} 
\end{itemize}
\item That's how I choose to adress them 
\begin{itemize}
\item Summary of research work
\end{itemize}
\textit{Or should that be integrated with the previous point? }
\end{enumerate}


\clearpage
\section*{Summary of publications and conferences}
\singlespacing
\textbf{Chapter 1 :  Bioeconomic Models for Terrestrial Social Ecological System Management : a Review}, S.Jean and L. Mouysset\\
\href{https://github.com/sim-jean/review-irere}{Replication code} and \href{https://zenodo.org/records/6656433}{data} are freely accesible
\begin{itemize}
\item Published in \textit{International Review of Environmental and Resource Economics}\\
 DOI : 10.1561/101.00000131
\item Presentations : 
\begin{itemize}
\item European Association of Environmental and Resource Economists (EAERE) Annual Conference, Rimini, 2022
\item ABIES Doctoral Days - Best Poster Award, 2022
\end{itemize}
\end{itemize}
%
\textbf{Chapter 2 : The Wildfire-Habitat Connectivity Dilemma: a Graph Theoretical Approach to Landscape Management}, S.Jean and L. Mouysset\\
\href{https://github.com/sim-jean/Landscape_connectivity_dilemma}{Replication code and data} are freely accessible
%
\begin{itemize}
\item Working paper
\item Presentations : 
\begin{itemize}
\item BINGO Seminar, CIRED, 2023
\item Interdisciplinary PhD in Sustainable Development, Columbia University, 2023
\end{itemize}
\end{itemize}
%
\textbf{Chapter 3 : Fences - the Economics of Movement in Mobile Public Bads}\\
\href{https://github.com/sim-jean/fences}{Replication code and data} are freely accessible
%
\begin{itemize}
\item Working paper
\item Presentations : 
\begin{itemize}
\item French Association of Environmental and Resource Economists, Université Savoie Mont-Blanc, 2024
\item Parisian PhD Seminar in Environmental Economics, Nogent sur Marne, 2024
\item CIRED Internal Seminar, 2024
\end{itemize}
\end{itemize}
%
\textbf{Chapter 4: Little downside and susbtantial gains result from farming of \textit{Totoaba Macdonaldi}}, J. Lawson, S.Jean (co-first authors), A. Steinkruger, M. Castellanos-Rico, G.M. Goto, M.A. Cisneros-Mata, E. Aceves Bueno, M.M. Warham, A.M. Sachs and S.D. Gaines\\
\href{https://github.com/julawson/conservation_farming_totoaba}{Replication code and data} are freely accessible.
%
\begin{itemize}
\item Under review at \textit{NPJ Ocean Sustainability}
\item Presentations:
\begin{itemize}
\item BIOECON Network Annual Conference, University of Santiago de Compostela, 2023
\item Trade and the Environment, Paris Saclay Applied Economics, 2023
\item European Association of Environmental and Resource Economists Annual Conference, University of Leuven, 2024
\end{itemize}
\end{itemize}
%
