\chapter*{Introduction}
\label{Introduction}

What should I say in the general introduction of my thesis? 

\begin{itemize}
\item Une page qui résume : le déclin de la biodiversité, l'économie bioéconomique, la contribution méthodologique de la thèse, et les résultats nouveaux. 

\item Introduction générérale thématique : le déclin de la biodiversité / 3-5 pages avec des graphiques. 
\begin{itemize}
\item Le déclin de la biodiversité de façon générale
\item Par ecosystème et taxon : on parlera alors de poissons, d'espèces invasives, de forêts
\item En rajoutant un petit quelquechose sur le changement climatique
\item Les causes sont à chercher du coté des hommes
\end{itemize}

\item D'où la necessité d'une approche par les sciences sociales, et l'économie peut y apporter beaucoup, elle l'a déja fait : social ecological systems, JEEM. 
\item Pourquoi l'économie et la modélisation bioéconomique?
\begin{itemize}
\item Un problème d'externalité global, de biens publics, de valeurs d'options, d'informations e.g. tous les problèmes spécifiques à l'économie de l'environnement
\item Un outil permettant la modélisation et la prospective: c'est à dire la description, sur base axiomatique, des comportements, à la fois individuels, non coopératifs, et de politique publique. 
\item Qui permet d'articuler grâce à un langage partagé différent champs de connaissance, notamment de dialoguer avec les sciences du vivant
\item Nonobstant les contributions
\item Il existe des champs d'application inexplorés, ou des questions importantes non résolues : bien lister ici les dimensions : marché, espace, politique publique/gestion privée. 
\end{itemize}

\item \textbf{ On a donc deux éléments bien identifiés dans la recherche et les questions qu'ils posent }: l'espace et la structure de marché. 
\item La structure de marché : 
\begin{itemize}
\item Question bien vieille dans la litérature sur les ressources naturelles: Solow, Hotelling etc
\item Moins bien tranchée sur la question des ressources renouvelables : rhinos (AER), foresterie etc ... 
\item On l'étudie donc dans un cas précis, c'est le chapitre 2
\end{itemize}
\item La question de l'espace : 
\begin{itemize}
\item L'espace est une dimension importante à prendre en compte, car il conditionne l'exploitation des ressources, autant que comment les décisions doivent être prises : ce qui change (Sanchirico et Costello)
\item C'est un challenge qui pose des questions de politique publique : la gestion de l'espace, dans un contexte de fragmentation, de prévention des risques etc est cruciale 
\item Il faut aussi comprendre comment l'espace, et les processus écologiques qu'il construit, sont formés par les individus [à raffiner] c'est le chapitre 3
\end{itemize}
\item Bien mentionner que l'intégralité des données, codes etc sont disponibles gratuitement et librement. 
\end{itemize}

\textbf{What's left to do}
\begin{itemize}
\item List contributions from the literature review, and find a way to put more into it. 
\item Find references and graphs from the institutions to document biodiversity loss
-> IPBES?
\end{itemize}

\clearpage
\section*{Summary of publications and conferences}
\singlespacing
\textbf{Chapter 1 :  Bioeconomic Models for Terrestrial Social Ecological System Management : a Review}, with L. Mouysset\\
\href{https://github.com/sim-jean/review-irere}{Replication code} and \href{https://zenodo.org/records/6656433}{data} are freely accesible
\begin{itemize}
\item Published in \textit{International Review of Environmental and Resource Economics} - DOI : 10.1561/101.00000131
\item Presentations : 
\begin{itemize}
\item European Association of Environmental and Resource Economists (EAERE) Annual Conference, Rimini, 2022
\item ABIES Doctoral Days - Best Poster Award, 2022
\end{itemize}
\end{itemize}
%
\textbf{Chapter 2: Little downside and susbtantial gains result from farming of \textit{Totoaba MacDonaldi}}, with J. Lawson (co-first author), A. Steinkruger, M. Castellanos-Rico, G.M. Goto, M.A. Cisneros-Mata, E. Aceves Bueno, M.M. Warham, A.M. Sachs and S.D. Gaines\\
\href{https://github.com/julawson/conservation_farming_totoaba}{Replication code and data} are freely accessible.
%
\begin{itemize}
\item Under review at \textit{NPJ Ocean Sustainability}
\item Presentations:
\begin{itemize}
\item BIOECON Network Annual Conference, University of Santiago de Compostela, 2023
\item Trade and the Environment, Paris Saclay Applied Economics, 2023
\item European Association of Environmental and Resource Economists Annual Conference, University of Leuven, 2024
\end{itemize}
\end{itemize}
%
\textbf{Chapter 3 : The Wildland Connectivity Dilemma : a Graph Theoretical Computational Approach}, with L. Mouysset\\
\href{https://github.com/sim-jean/Landscape_connectivity_dilemma}{Replication code and data} are freely accessible
%
\begin{itemize}
\item Working paper
\item Presentations : 
\begin{itemize}
\item BINGO Seminar, CIRED, 2023
\item Interdisciplinary PhD in Sustainable Development, Columbia University, 2023
\end{itemize}
\end{itemize}
%
\textbf{Chapter 4 : Fences - the Economics of Movement in Mobile Public Bads}\\
\href{https://github.com/sim-jean/fences}{Replication code and data} are freely accessible
%
\begin{itemize}
\item Working paper
\item Presentations : 
\begin{itemize}
\item French Association of Environmental and Resource Economists, Université Savoie Mont-Blanc, 2024
\item Parisian PhD Seminar in Environmental Economics, Nogent sur Marne, 2024
\item CIRED Internal Seminar, 2024
\end{itemize}
\end{itemize}
%

