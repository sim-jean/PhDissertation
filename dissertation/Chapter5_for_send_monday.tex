\chapter{Fences - The economics of endogenous connectivity in spatial renewable bads}
%\footnote{I am grateful to Lauriane Mouysset and Christopher Costello, Martin Quaas, Giorgio Fabbri, Francesco Ricci, Valentin Cocco, Jérôme Pivard, Lucas Vivier, Romain Fillon, Alexandre Adrian (from Platt Vineyard),  as well as seminar participants at BINGO, the 2024 FAERE PhD Workshop, 2024 Parisian PhD Seminar in Environmental Economics for their helpful comments.}}

%To fragment or thin : 


\begin{minipage}{0.9\textwidth}
\singlespace
%This article discusses the complex interplay between ecological conservation and economic activities within spatially distributed networks. Renewable ressources, such as game, and invasive species on land can be managed through their stock, and landscape connectivity. This article examines how overharvesting (or under in the case of an invasive species) and habitat fragmentation can be jointly managed. Spatial connectivity can be managed to solve the tragedy of the commons and lead to efficient harvest as property rights are secured. Nonetheless, fragmenting habitat can be detrimental, and a trade-off emerges between efficient harvesting and optimal fencing. This article examines the optimal management of spatially distributed, endogenously connected renewables. In a setting where economic and biological conditions are heterogeneous and positively correlated, efficient management leverages a spatial arbitrage opportunity, promotes habitat connectivity, and optimizes welfare. With negative correlations, efficient management implies not changing what Nature has done, and harvest rules need to be enforced for welfare maximization. On the other hand, I look at the dynamic game that emerges when players can choose their harvest level and how their land connects to others. I show, under mild conditions, that the non-cooperative equilibrium results in fragmented habitat but halts overexploitation. In some cases, it can result in optimal management, depending on the underlying heterogeneous economic and biological conditions. Moreover, observed ecosystem dynamics can be partly rationalized through economic factors. For instance, observed sink-source dynamics may reflect heterogeneous biological productivities, and environmental features, but also heterogeneous returns to species in non-cooperative equilibria. 
%Finally, I  examine which policies should be implemented in a second-best framing, where connectivity or harvesting policies can be implemented, and rank policy options across heterogenous landscapes. 
%This article contributes to the literature on optimal renewable management and brings the fishery literature back on land, as connectivity can be managed. It contributes to the dynamic network formation games literature in the context of a renewable resource. I emphasize the need for policies that consider the interconnectedness of ecological systems and the different margins of economic activities that impact them, aiming for sustainable management practices that ensure both ecological integrity and economic viability.
%It explores how optimal resource management changes with endogenous connectivity. Moreover, as connectivity can be changed, the extent of spatial externalities can be limited. This article highlights conditions where solving the spatial externality is the first best scenario instead of leveraging what Nature has rightfully done. 

This article examines the management of spatially distributed renewable resources—specifically wildlife and infectious diseases—through the lens of economic and spatial analysis. I focus on "bads" like invasive species and diseases, which cause economic and ecological harm, and utilize population control and fencing as central mechanisms. I analyze how fencing influences resource flow and connectivity. On the one hand, in the presence of ecological and economic heterogeneities, fencing can be used to leverage spatial artbitrage opportunities. On the other hand, while promoted as a tool to incentivize the internalization of costs associated with ``bads", they may undo what Nature has rightfully done. In this sense, while fencing may be welfare improving in a setting with initially poor connectivity, an uncoordinated use of fencing, although welfare improving, is not welfare maximizing. The study extends a theoretical model \citep{costello_private_2017} that integrates aspects of stock and patch connectivity management and explores both cooperative and non-cooperative management strategies. The findings indicate that optimal management often requires a nuanced understanding of the spatial dynamics and economic costs associated with different control strategies. We present a series of propositions that characterize the conditions under which fencing and resource control strategies can be optimized, including the interaction effects of exclusionary and trap effects. This article contributes to the literature by highlighting the role of spatial heterogeneity in the management of renewable resources and providing insights into the formulation of more effective environmental policies, as it analyzes how to design policies on a subset of the landscape, to maximize economic and ecological benefits. \\\\
\textit{JEL codes :} Q20, Q24, R12\\
\textbf{Keywords :} spatial resource management, invasive species; fencing and control strategies; optimal management; non-cooperative equilibrium; second-best policy.
\end{minipage}


\clearpage

{\footnotesize
\bibliographystyle{abbrvnat}
\bibliography{bibliography_these}
}