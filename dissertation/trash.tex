

\clearpage
%, at various conceptual levels. First, it requires understanding the direct consequence of each subsystem onto the other : how do economic forces shape ecosystems? How do ecosystems shape economic forces? To answer these questions, conflicting links may be acknowledged and give rise to trade-offs
Ce qui est intéressant dans ce qui précède, c'est la mention de l'interplay entre différentes dimensions. 
On peut parler de l'interplay entre différentes dimensions à l'intérieur d'un même système, reliées par un lien dans le système environnemental. 
On peut ensuite regarder la typologie de ces interactions : 


Dans ma thèse, je fais quoi : 
\begin{itemize}
\item Un papier sur les poissons:
\begin{itemize}
\item Contribution thématique : est-ce qu'un monopole est bien pour les ressources? Est-ce que l'aquaculture en duopole c'est bien? Est ce que le farming c'est bien? 
\\
Les trucs qui émergent ici c'est donc : est-ce que la provision d'un\textbf{ substitut au naturel} permet au naturel d'être moins exploité? ; comment les institutions \textbf{que l'on design permettent d'améliorer l'interface entre humain et nature}; est-ce que restreindre l'accès à un marché, quand bien même c'est pas terrible pour les consommateurs, c'est un moyen de préserver la ressource; quels sont les designs institutionnels qui permettent de mettre au diapason les dynamiques? Les deux questions sont : selon quel mode le rapport entre nature et humain fonctionne-t-il (extraction, fourniture, substitution ou soustraction) et comment les arrangements institutionnels, de façon générale, influe sur la ressource (comment l'accès au marché, et l'octroi de droits \textit{de facto} plutôt que \textit{de jure} influence la ressource? Comment les droits sur un substitut influencent la ressource, dans un cas où il est difficile de faire \textit{respecter le droit}?
\\
Ca, ça fonctionne pour une espèce endémique, pour un écosystème bien précis, dans le cadre d'une interaction homme nature identifiée, avec des co bénéfices; pas vraiment une approche systémique, un \textbf{seul objectif}
\item Contribution méthodologique : analyse des structures de marché dans l'exploitation des ressources naturelles
\end{itemize}
$\Rightarrow$ Deux idées fortes : le design des interfaces de marché entre humain et nature est clé, lorsque l'on parle d'exploitation des ressources dans un context où le droit est difficilement respectable, et la concentration potentiellement la norme : on le sait mais pas non plus full exploré. On a aussi des évidences sur l'impact de la substituabilité de ressources humaines à du non humain (eyal frank), mais on en sait moins sur l'impact potentiel dans l'autre sens. 


\item Le papier sur les forêts : climate change is also an important driver, need to highlight that in the motivation part; \\
Forest ecosystems are a typical example of multiple interplays : economic value to protect, economic value destroys it; provide amenity, but also foster risk, or bear risk; provide habitat, but potential risk; \\
Economic drivers are collective and individual, we want to safeguard biodiversity but also to reduce the risk we face, especially as these risks become uninsurable : problem, we need to find adaptation paths to mitigate the biodiversity crisis, and find ways that satisfy both objectives? 
\\
La grande idée là c'est quoi? C'est que les propriétés physiques d'un écosystème donnent lieu à différents phénomènes, que l'on veut et que l'on ne veut pas. On est confrontés à des écosystèmes dont on doit finement gérer les différentes implications. On a \textbf{plusieurs objectifs au sein d'un ecosystème}. Bon d'accord, et une fois qu'on a dit multi objectif, c'est quoi le délire? C'est pas que ça le problème, il faut dire quelque chose sur la dimension économique. Non assurable $\Rightarrow$ voir les politiques publiques, comment gérer ça? Pourquoi c'est frappant? Au delà de la méthode? 
\\
Est ce que l'idée c'est qu'il faut intégrer l'espace, la construction de l'espace dans l'analyse économique? Que l'espace c'est un construit, et que au delà des flux spatialisés, il faut analyser la manière dont on influence la formation de l'espace, et de ses propriétés? Trouver des angles d'analyses où la formation même de l'espace est au 
\\
Idée clé ici c'est que l'espace et ses caractéristiques donnent lieu à des conséquences positives et négatives. Etant donné le réchauffement climatique on the one hand et le déclin de la biodiversité on the other hand, il faut trouver des moyens d'aménager l'espace qui nous permettent à la fois de nous adapter et de mitiger le déclin; La multifonctionalité des écosystèmes et les externalités spatiales qu'ils génèrent doivent nous pousser à manager ces externalités, en (i) en choisissant l'extant dans un contexte où leurs directions sont fixes mais contingentes (e.g. le papier sur les feux) et (ii) en choisissant l'extant et les directions dans un contexte où l'espace est modulable. 

\item Enfin, le dernier papier, c'est une idée simple : on joue nous même sur la connectivité. Elle crée une externalité spatiale, du fait des ressources qui bougent; on voit qu'une propriété naturelle, qui n'en est pas qu'une 

\end{itemize}




%One of the key elements is to identify the various uses and values carried by ecosystems within social-ecological systems, including conflicting perspectives\\
%From section \ref{intro:facts}, curbing the 
\textbf{More challenges, why it is interesting, big picture, to study that, in words, what are the challenges in terms of thought and why we should care, in terms of economics and big picture ideas.}


%As \textit{sustainable} practices imply a collective viewpoint, shifting away from individual decision-making, it also implies a global ecological perspective, acknowledging the interplay between social ecological system components, and the multiple functions some ecosystem underlie in social-ecological systems.
% Practices for sustainability and Aichi targets

\begin{itemize}
\item List of thematic challenges and how case studied - \textbf{link with Aichi and 30 by 30?}
\begin{itemize}
\item The first order of priority is to curb land use change for terrestrial ecosystems, to halt overexploitation and invasive species (so increase control taking space into account)
\item Also to find ways to increase habitat, while measuring the risks it poses on other aspects, in a context of climate change. 
\item Find ways to curb IUU and poaching, in difficult context where economic value is tied to the weak governance : case study of totoaba? 
\end{itemize}
\item List of big ideas that outlines the contributions : 
\begin{itemize}
\item Wildland connectivity : we need to increase the connectivity of wildlands, but also not too much; akin to managing a portfolio of resources with correlated risks?; we also want to maintain a certain level of performance for ecological criteria, and need public policy to do so.
\item Finally, as food from the sea provides X\% of global protein, aquaculture is a lever to alleviate fishing pressure and provide food and protein intake globally, as well as value for local communities. 
\end{itemize}
\end{itemize}

\section*{Bioeconomic modeling : a framework to analyze the biodiversity crisis}

As highlighted in chapter \ref{chapter1}, bioeconomic modeling has been widely used to understand the economic forces that drive resource overexploitation in terrestrial and marine settings for threatened species, the ways to address invasive species and their damages in forest landscapes, as well as ways to optimally conserve species on landscapes with varying degrees of human intervention, ranging from \textit{wilderness} to agricultural landscapes. However, this study also highlighted methodological shortcomings in bioeconomic modeling (along with \cite{Drechsler20200}) that are yet to be fully included in the  bioeconomic toolbox to help solving the biodiversity crisis. Among them, the full interplay of stochastic processes in natural and economic phenomena\footnote{While the natural resource maganagement literature has examined how risk affects decision making with risk neutral perspectives \citep{reed_1979_optimal, costello_optimal_2008}, risk and tipping \citep{costello_renewable_2019} and risk averse perspectives \citep{McGoughPlantingaCostello+2009,kapaun_does_2013,TAHVONEN2018659}, the full effect of different attitudes towards risk and consumption smoothing is a recent endeavor. Disentangling the effect of risk and time preferences, \citep{quaas_2019_insurance, AugeraudVeron2019} characterize the insurance value of capital. Recently, \citep{KELSALL2023102855} characterize the effect of preferences towards risk and intetemporal variability of income on resource extraction.}, the inclusion of participatory approaches, 
% List the shortcomings from the literature : not so sure about that. 

\clearpage
First, the full extent of natural and economic stochasticity still remains to be studied as it is a burning issue in the context of climate change and increased environmental stochasticity.
   Among other 
  
  on resource extraction has only recently been studied (Kelsall). The relative effect of risk aversion and intertemporal income variability has been to showcase the insurance role of biological diversity, but has yet to be fully analyzed   disentangling the role of different sources of risk, temporal consumption smoothing and intertemporal variability has highlighted insurance values of biodiversity, but only recently been studied  (see 


Among these dimensions, I chose to focus on the roles of habitat connectivity and market structure, as they are key to understand and adress the two most important drivers of global biodiversity decline, land/sea use change and overexploitation of resources. 

two components are key to adressing major drivers of biodiversity decline



% Reprendre cela : 
Stochasticity governs ecological and economic processes, and has gradually been included in decision-making from resource managers, and how risks affect resource extraction (Reed, Costello, Costello et al), role of multiple risks together is being studied (Costello et al), and how dynamic choices, uncertainties etc are being considered (check intro by Claudia). 
% 
\clearpage


What should I say in the general introduction of my thesis? 

\begin{itemize}
\item Une page qui résume : le déclin de la biodiversité, l'économie bioéconomique, la contribution méthodologique de la thèse, et les résultats nouveaux. 

\item Introduction générérale thématique : le déclin de la biodiversité / 3-5 pages avec des graphiques. 
\begin{itemize}
\item Le déclin de la biodiversité de façon générale
\item Par ecosystème et taxon : on parlera alors de poissons, d'espèces invasives, de forêts
\item En rajoutant un petit quelquechose sur le changement climatique
\item Les causes sont à chercher du coté des hommes
\end{itemize}

\item D'où la necessité d'une approche par les sciences sociales, et l'économie peut y apporter beaucoup, elle l'a déja fait : social ecological systems, JEEM. 
\item Pourquoi l'économie et la modélisation bioéconomique?
\begin{itemize}
\item Un problème d'externalité global, de biens publics, de valeurs d'options, d'informations e.g. tous les problèmes spécifiques à l'économie de l'environnement
\item Un outil permettant la modélisation et la prospective: c'est à dire la description, sur base axiomatique, des comportements, à la fois individuels, non coopératifs, et de politique publique. 
\item Qui permet d'articuler grâce à un langage partagé différent champs de connaissance, notamment de dialoguer avec les sciences du vivant
\item Nonobstant les contributions
\item Il existe des champs d'application inexplorés, ou des questions importantes non résolues : bien lister ici les dimensions : marché, espace, politique publique/gestion privée. \\
From the review:
\begin{itemize}
\item Stochasticity : how stochasticity governs ecological and economic processes is a recent strand, but studied (costello, augereau véron, quaas and Baumgartner). Remain an open question as to how do people respond to uncertain and risky contexts, with the limited possibility of insurance; mentionner des choses à la Romain, genre risques reliés : stochasticité de la ressource écologique et incertitude autour de la détection des invasive species etc. 
\item Community perspective for ecosystem based conservation planning, e.g. interaction between several species at the landscape level
\item The role of spatial processes, for movement, different habitat quality, understanding how space changes results in combination with other determinants of bioeconomic models (Sanchirico, Costello as well)
\item Intricate property rights, ranging from the most localy complex in terms of types of interactions (externality locally), local competition for resource, different levels of political management, but also different types of market structure

\item Incorporating different value types, including indigenous knowledge, in bioeconomic modeling for policy making

\item incorporating more evidence from quasi experimental methods in the results
\end{itemize}
\end{itemize}

\item \textbf{ On a donc deux éléments bien identifiés dans la recherche et les questions qu'ils posent }: l'espace et la structure de marché. 
\item La structure de marché : 
\begin{itemize}
\item Question bien vieille dans la litérature sur les ressources naturelles: Solow, Hotelling etc
\item Moins bien tranchée sur la question des ressources renouvelables : rhinos (AER), foresterie etc ... 
\item On l'étudie donc dans un cas précis, c'est le chapitre 2
\end{itemize}
\item La question de l'espace : 
\begin{itemize}
\item L'espace est une dimension importante à prendre en compte, car il conditionne l'exploitation des ressources, autant que comment les décisions doivent être prises : ce qui change (Sanchirico et Costello)
\item C'est un challenge qui pose des questions de politique publique : la gestion de l'espace, dans un contexte de fragmentation, de prévention des risques etc est cruciale 
\item Il faut aussi comprendre comment l'espace, et les processus écologiques qu'il construit, sont formés par les individus [à raffiner] c'est le chapitre 3
\end{itemize}
\item Bien mentionner que l'intégralité des données, codes etc sont disponibles gratuitement et librement. 
\end{itemize}

\textbf{What's left to do}
\begin{itemize}
\item List contributions from the literature review, and find a way to put more into it. 
\item Find references and graphs from the institutions to document biodiversity loss
-> IPBES?
\end{itemize}
\clearpage

\section{New intro}
\subsection{Plan global}
Idée d'organisation : 
\begin{enumerate}
\item Documenter la crise de la biodiversité : état et causes\\
Un point de plus sur l'habitat, les poissons, et les usages des forêts
\item La nécessité d'une trajectoire soutenable : définition de la soutenabilité, actions à engager, afin d'avoir un usage soutenable de la nature
\item Un cadre d'analyse pour les penser
\item Les limites existantes
\item Comment les surmonter, et appliquer cela aux cas qui m'intéressent?
\end{enumerate}
\subsection{Plan détaillé}

 
 



\begin{enumerate}
\item Lay the facts : global biological diversity decline (1-2 pages)
\begin{itemize}

\item An overall decline over time : 
\begin{itemize}
\item Across taxa : for types of animals, document extinctions across taxa and decline in populations
\item Across ecosystems : terrestrial and marine, forests etc
\end{itemize}

\item What is biological diversity? How can we define it? What values does it call for? \textit{Est ce que ça a vraiment sa place ici?} - peut être plutôt note de bas de page, sur les difficultés du concept etc, mais on se dit richness and abundance across the world

\item The causes are man made, on land and at sea : land use change, overexploitation, habitat destruction and fragmentation, as well as climate change (for wildfire)
\\
Notion of direct and indirect drivers\\

Rajouter une bonne couche sur la nécessité de (i) pas trop exploiter et pourquoi; et (ii) sur la fragmentation de l'habitat (avec pas mal de concepts écologiques, genre surface, fragmentation, stepping stones etc)
\item Weak substitutability of biological diversity.\\
\item International and global policy goals in IPBES, Biodiversity Strategy for 2050 etc, pour ancrer les objectifs de la thèse.
\end{itemize}

\item Lik with substitutability, and need for sustainable approaches : can use the table SPM1

\item Need a unifying method, conceptual body, to remedy this crisis : find pathways towards a sustainable future etc\\
$\Rightarrow$ Bioeconomic modeling : what it is

\begin{itemize}
\item What it does : mix together ecological dynamics and economic drivers
\item Strengths : process based, out of sample, policy design
\item Applied to various ecosystems : forestland a lot, oceans and fisheries, and agricultural land to have several \\
Maybe list some results?
\end{itemize}

\item The current state of biodiversity calls for an ``ecosystem centered'' management\\
\textit{Unconfortable with that, but it allows to introduce bioecon modeling, and highlight the shortcomings based on empirical facts}
\begin{itemize}
\item Ecosystems feature different uses, different species, offer different risk and benefits etc : cases of forests (risks of wildfire, habitat to biodiversity, forestry industry, leasure); and oceans (fisheries, conservation of habitat, leasure).
\item The literature has gradually evolved towards encompassing the many dimensions of ecosystem based management, but shortcomings remain : 
\begin{itemize}
\item Stochasticity : how stochasticity governs ecological and economic processes is a recent strand, but studied (costello, augereau véron, quaas and Baumgartner). Remain an open question as to how do people respond to uncertain and risky contexts, with the limited possibility of insurance; mentionner des choses à la Romain, genre risques reliés : stochasticité de la ressource écologique et incertitude autour de la détection des invasive species etc. 
\item Community perspective for ecosystem based conservation planning, e.g. interaction between several species at the landscape level
\item The role of spatial processes, for movement, different habitat quality, understanding how space changes results in combination with other determinants of bioeconomic models (Sanchirico, Costello as well)
\item Intricate property rights, ranging from the most localy complex in terms of types of interactions (externality locally), local competition for resource, different levels of political management, but also different types of market structure

\item Incorporating different value types, including indigenous knowledge, in bioeconomic modeling for policy making
IPLCs have been using fire to promote herbaceous vegetation
and useful game or plant species (Pechony \& Shindell,
2010; Valladares et al., 2014). Such historical practices and
other land-use legacies combined with more recent driving
forces, such as land abandonment and fire suppression
strategies, have been playing a major role in reshaping the
Mediterranean landscapes (Blondel, 2006; Gauquelin et al.,
2018; Marlon et al., 2008; Valladares et al., 2014).

\item incorporating more evidence from quasi experimental methods in the results
\end{itemize}
\end{itemize}
\item In this dissertation, I chose to focus on space and market structure across 2 different types of ecosystems
\begin{itemize}
\item Space \& Market structure : make up policy, conceptual and methodological challenges, as they adress the principal causes of global biodiversity decline and foster technical and methodological challenges, as highlighted by our review and other. 
\item Space on terrestrial landscapes is important to tackle main causes of species decline: 

\begin{itemize}
\item Classical arguments : featuring space in traditional models allows to understand better aggregate behavior (Sanchirico); externalities arise from spatial dependence, and optimal behavior can be characterized in some cases (Costello)

\item Spatial dependence creates problems, but landscape \textbf{connectivity} is of the essence : 
\begin{itemize}
\item Hu et al
\item Tischendorf \& Farig
\item May et al
\item Fahrig
\item Hanski
\item Fischer \& Lindenmayer
\end{itemize}

\item Forests support a variety of functions and risks, benefits and costs; these risks and benefits have distinct spatial patterns, and occur at a large scale, calling for an aggregate scale approach; different phenomena are characterized by different scales, different time frames and different ??? but connectivity is an important feature that structures ecological processes; raises specific challenges with spatial optimization; spatial management of connectivity with multiple use/functions ecosystems is complicated, but useful in terms of policy and methods\\
Additionally, in a context of resource scarcity, space can be a driver of policy success for both resource preservation and wildfire prevention. 
\item Spatial heterogeneity drives movement; spatial ressources are difficult to manage as they create spatial externalities, that are dynamic through time; it has long been considered that movement is an exogenous process, or driven exclusively by ecological features; besides fragmentation, as well as corridors etc, human actions at a more local scale can change the pattern of spatial movement, especially in the case of invasive species. Petite envolée sur le fait que la géographie est le résultat d'un processus social? When possible, people may fence, to resolve the externality; in doing so, they may have a better use of the resource (e.g. halt overexploitation) but undermine landscape connectivity (e.g. increase fragmentation); it is shown that not all fences are bad, and connectivity is good if heterogeneity exists; 

\item Leverages specific challenges when dealing with dynamics : dimensionality curse, which gives rise to several types of methods to solve the problem, when complexity dimensions can be trimmed (chapter on connectivity uses a simplified dynamic with a lot of space and state dependence, while fences uses a lot of space, non simplified dynamics, but with no state dependence)
\end{itemize}

\item Market structure matters for resource exploitation

\begin{itemize}
\item In the harvesting literature, it is often considered that marginal revenue is constant as prices are determined by an international market. Additionally, open access is a key feature of the commons, and a small group of people seldom controls a complete market.
\item However, some specific cases of endemic resources, such as wildlife trade, in informal contexts, can be characterized by restricted access
\item A monopolist may be a conservationist's best friend : intuitively, restrict quantities to maximize profits. This is dependent on market specifics, discounting etc (Hanesson)
\item Additionally, competition in intricate settings can trigger weird dynamic, according to existing papers 
\end{itemize} 
\end{itemize}
\item That's how I choose to adress them 
\begin{itemize}
\item Summary of research work
\end{itemize}
\textit{Or should that be integrated with the previous point? }
\end{enumerate}

\clearpage